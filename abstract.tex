\documentclass[sigplan,natbib=false]{acmart}
\usepackage{inputenc}
\usepackage[]{biblatex}
\bibliography{bib.bib}
% \citestyle{acmauthoryear}
\acmConference[CoqPL25]{The Eleventh International Workshop on Coq for Programming Languages}{Sat 25 Jan 2025}{Denver, Colorado, USA}
\acmYear{2025}

\author{Enrico Tassi}
\orcid{0000-0002-7783-528X}
\affiliation{%
    \institution{Université Côte d'Azur, Inria}
    %\city{bla}
    \country{France}
}
\email{enrico.tassi@inria.fr}
\acmDOI{}

\title{Elpi: rule-based meta-languge for Rocq}
\begin{abstract}
% Elpi is a high level programming language that can be used to implement new
% commands and tactics for the Rocq prover. It features native support for
% syntax trees with binders and holes, freeing the programmer from the complexity
% of De Bruijn indexes and unification variables.
% 
% In recent years Elpi was used to write a number of Rocq extensions, some
% of which are widely used. In this talk we give a gentle introduction to the
% programming language Elpi and its extensive API to interact with Rocq; 
% we survey some applications written in Elpi; finally we compare it to other
% meta-languages discussing its strong and weak points.

Elpi is a high-level programming language designed to implement new commands
and tactics for the Rocq prover. It provides native support for syntax trees
with binders and holes, relieving programmers of the complexities associated
with De Bruijn indices and unification variables.

In recent years, Elpi has been used to develop a variety of Rocq extensions,
some of which have become widely adopted. In this talk, we will provide a
gentle introduction to the Elpi programming language and its extensive API for
interacting with Rocq. We will also survey notable applications written in Elpi
and conclude by comparing Elpi to other meta-languages, highlighting both its
strengths and weaknesses. \end{abstract}

\begin{document}
\maketitle

\section{Elpi = $\lambda$Prolog + C.H.R.}

% From a programming language perspective Elpi is a dialect of
% $\lambda$Prolog enriched with constraints and constraint handling rules.
% From a more practical angle Elpi is an extension language: an interpreter
% that can easily hosted by a larger application and that can provide users a high
% level language to extend the application.
% 
% Elpi and its implementation provide a unique mix of feature that make it
% particularly well suited to extend an interactive theorem prover like Rocq.

From a programming language perspective, Elpi is a dialect of $\lambda$Prolog
enriched with constraints and constraint handling rules. From a more practical
standpoint, Elpi is as an extension language: an interpreter that can be easily
embedded within a larger application, providing users with a high-level
language to extend the application's functionality.

Elpi and its implementation provide a unique combination of features that make
it particularly well-suited for extending an interactive theorem prover like
Rocq.

\paragraph{Rule based}
% Elpi source code is organized into rules, and new rules
% can be added both statically and dynamically, either by the programmer or
% by the program itself. This feature goes hand in hand with proof scripts
% that similarly build an ever growing environment of types and proofs. New
% rules can mention new concepts and hence extend the capabilities of existing
% Elpi programs.

Elpi source code is organized into rules, which can be added both statically
and dynamically, either by the programmer or by the program itself. This
feature aligns naturally with proof scripts, which similarly construct an ever
growing environment of types and proofs. New rules can use new concepts and
hence extend the capabilities of existing Elpi programs.

\paragraph{Syntax trees with binders}
Elpi data types can contain binders and the programming
language offers primitives to cross then, following the Higher Order Abstract
Syntax tradition~\cite{10.1145/960116.54010}.
In particular the programmer can
postulate fresh constants and substitute bound variables with them,
a technique called binder mobility~\cite{miller:hal-01884210}.

\paragraph{Context management}
% When a binder is crossed it is often required to
% attach some data to the bound variable. For example typing algorithms
% thread a typing context. When written in Elpi these algorithms do not need to
% do mention a context and can rather piggy back on the runtime of the programming
% language by adding new rules dynamically.

When a binder is crossed, it is often necessary to associate some data with the
bound variable. For example, typing algorithms typically manage a typing
context. In Elpi, these algorithms do not need to explicitly handle a context;
instead, they can leverage the runtime of the programming language by
dynamically adding new rules.

\paragraph{Unification variables}
% Being a logic programming language Elpi
% provides unification variables, carries a substitution for the programmer
% and does scope checking preventing variable capture.

As a logic programming language, Elpi provides unification variables,
automatically manages substitutions for the programmer, and performs scope
checking to prevent variable capture.

\paragraph{Constraints}
%A constraint is a computation that is suspended
%until its input, a logic variable, materializes. Constraint handling rules manipulate
%suspended computations as first class values, and deduce new information.
%For example if is X is an unknown number on which two incompatible
%computations are suspended, like the tests for being even and odd, then a
%constraint handling rule can immediately stop the program.

A constraint in Elpi represents a computation that remains suspended until its
input, a unification variable, becomes instantiated. Constraint Handling Rules (CHRs)
treat these suspended computations as first-class values,
enabling them to deduce new information dynamically.
For example, if \texttt{X} is an unknown natural number on which two incompatible
computations, such as tests for being even and odd, are suspended,
a CHR can detect the conflict and halt the program.

\paragraph{Syntax trees with holes}
% Logic variables can be used to represent
% incomplete syntax trees. Algorithms manipulating this kind of trees need
% to attach data to holes exactly as typing algorithms need to attach data to
% bound variables. For example if the syntax tree being built is a Rocq term, then
% these holes surely come with a typing constraint.
% Whenever a hole gets instantiated the typing constraint must be checked for
% validity. Moreover, since the same hole cannot have two types, a constraint
% handling rule can enforce the unification of two typing constraints about the same hole.

Unification variables in Elpi can be used to represent incomplete syntax trees.
Algorithms that manipulate these trees must attach data to holes,
similar to how typing algorithms attach data to bound variables.
For instance, if the syntax tree being constructed represents a Rocq term,
these holes will certainly carry typing constraints. 

Whenever a hole gets instantiated, its typing constraint must be validated.
Additionally, since a hole cannot have two conflicting types, a CHR can enforce
the unification of any typing constraints associated with the same hole,
ensuring consistency throughout the process.

\paragraph{API}
% The Elpi interpreter comes with a very flexible Foreign Function
% Interface. For example it is possible to invoke the Rocq type checker deep
% under binders, or script the vernacular language by programmatically
% declaring inductive types, modules, type class instances, Arguments directives,
% and so on.

The Elpi interpreter includes a highly flexible Foreign Function
Interface. For example, it allows invoking the Rocq type checker
deep under binders or scripting the vernacular language by
programmatically declaring inductive types, modules, type class
instances, \texttt{Arguments} directives, and more.

\section{Applications}

We briefly describe some applications written in Elpi.

\paragraph{Hierarchy Builder}
HB~\cite{cohen_et_al:LIPIcs.FSCD.2020.34} is a high level language to describe hierarchies
of interfaces and is adopted by many Rocq libraries including the Mathematical
Components one~\cite{affeldt:hal-03463762}. HB synthesizes all the boilerplate in order to
make these interfaces work in practice, like modules, records, canonical
structure instances, implicit arguments declarations, notations. 
% It takes
% advantage of the vast API Elpi provides and the possibility to incrementally
% build a data base of known interfaces and their inheritance relation.
HB leverages the extensive API provided by Elpi, as well as its ability to incrementally
build a database of known interfaces and their inheritance relations.

\paragraph{TnT}
% Trakt and its successor Trocq~\cite{10.1007/978-3-031-57262-3_10} are
% frameworks to transport
% types (i.e. Rocq goals) over type (iso)morphisms, with or without univalence.
% The former is used by the Sniper tactic of Rocq-smt~\cite{DBLP:conf/cpp/Blot0CPKMV23}.
% Both tools take advantage of the capability of Elpi to manipulate syntax
% with binders. Trocq uses constraints to accumulate knowledge during
% term processing and finally compute an optimal solution.

Trakt and its successor Trocq~\cite{10.1007/978-3-031-57262-3_10} are
frameworks designed to transport types (Rocq goals) over type
(iso) morphisms, with or without univalence. The former is used by the Sniper
tactic in Coq-smt~\cite{DBLP:conf/cpp/Blot0CPKMV23}. Both tools leverage
Elpi's ability to manipulate syntax with binders. Trocq, in particular, uses
constraints to accumulate knowledge during term processing and ultimately
computes an optimal solution.

\paragraph{Derive}
% Derive synthesizes code from type declarations, like deep
% induction principles~\cite{tassi:hal-01897468},
% equality tests~\cite{gregoire:hal-03800154}, parametricity relations,
% lenses for record updates, etc.
% Derive is an extensible framework: each derivation is a set of rules and can
% depend on the result of other derivations. The formal methods team at BlueRock
% security extended it to cover the synthesis of notions of the Std++
% library\footnote{\url{https://github.com/bedrocksystems/BRiCk}}.

Derive synthesizes code from type declarations, such as deep induction
principles~\cite{tassi:hal-01897468}, equality
tests~\cite{gregoire:hal-03800154}, parametricity relations, lenses for record
updates, and more. Derive is an extensible framework, where each derivation is
defined by a set of rules that can depend on the results of other derivations.
The formal methods team at BlueRock Security has extended this framework to
cover the synthesis of concepts from the Std++ library.

\paragraph{NES}
NES emulates name spaces on top of Rocq module system. Unlike modules a namespace
is never closed and new items can be added to in, even in different files. NES
takes advantage of the capability of Elpi to script the Rocq vernacular
language.

\paragraph{Algebra-tactics} AT
% is a frontend to the ring, field, lra, nra, and psatz
% tactics that supports the Mathematical Components algebraic
% hierarchy~\cite{sakaguchi:LIPIcs.ITP.2022.29}.
% These tactics take advantage of the seamless interface of Elpi with Rocq
% by repeatedly calling Rocq's deep inside terms in order to reify expressions
% up to a controlled form of conversion.
is a frontend to the ring, field, linear real arithmetic (lra),
nonlinear real arithmetic (nra), and psatz tactics, designed to support the
Mathematical Components algebraic
hierarchy~\cite{sakaguchi:LIPIcs.ITP.2022.29}. These tactics leverage Elpi's
seamless integration with Rocq, repeatedly invoking Rocq's unification mechanisms
deep inside terms to reify expressions up to a controlled form of conversion.

\section{Related work}

Elpi is in good company.

\paragraph{
%$\mathcal{L}$tac} is the legacy extension language for Rocq~\cite{10.5555/1765236.1765246}. Its use
%in new projects is going to fade away, but there is a substantial code
%base in the wild using this language. In spite of a syntax close to
%functional programming language its semantics is closer to a logic programming
%language (i.e. the runtime implements backtracking). While Rocq terms
%can be manipulated using their natural syntax, i.e. no De Bruijn indexes,
%the language suffers from
%an unclear binding discipline, and sometimes confuses the variables of the
%meta language with the ones of the object language. The dynamic semantics
%is also sometimes surprising, for example t and idtac;t are very different,
%the latter being a thunk. The language has no algebraic data types and is untyped.
%In spite of these shortcoming the language has been,
%in our opinion, very instrumental to the success of Rocq.
$\mathcal{L}$tac} $\mathcal{L}$tac (version 1) is the legacy extension language for
Rocq~\cite{10.5555/1765236.1765246}. While its use in new projects is expected
to decline, a substantial codebase still relies on this language. Despite
having a syntax resembling that of functional programming languages, its
semantics is closer to those of a logic programming language, as the runtime
implements backtracking. Rocq terms can be manipulated using their natural
syntax—without the need for De Bruijn indices—though the language suffers from
an unclear binding discipline. This sometimes leads to confusion between the
variables of the meta-language and those of the object language. The dynamic
semantics of $\mathcal{L}$tac can also be surprising; for example, \texttt{t} and
\texttt{idtac; t} are very different, with the latter being a thunk.
Furthermore the language is untyped and lacks even basic data types such as lists.
Despite these shortcomings $\mathcal{L}$tac has been instrumental in the success of Rocq.

\paragraph{
% $\mathcal{L}$tac2} is an ML like language that hides in its runtime
% the proof engine monad of Rocq~\cite{10.1145/1090189.1086390}. Rocq terms are
% represented with an algebraic data type faithful to the internal Rocq term data
% type, including De Bruijn indexes. It features a notion of quotations and
% anti-quotations allowing to use Rocq term syntax without confusion. It exposes
% many APIs of the proof engine making it well suited to program low level tactic
% code, to the benefit of the efficiency of the code. It lacks completely APIs to
% declare Rocq inductive types and definitions, or script the vernacular.
$\mathcal{L}$tac2} The second version of $\mathcal{L}$tac is an ML-like language that encapsulates the proof engine
monad of Rocq~\cite{10.1145/1090189.1086390}. Rocq terms are represented using
an algebraic data type that closely mirrors the internal Rocq term data
structure, including De Bruijn indices. The language introduces a notion of
quotations and anti-quotations, which allows users to employ Rocq term syntax
without ambiguity. Additionally, it exposes many APIs of the proof engine,
making it well-suited for programming low-level tactic code and optimizing
efficiency. Currently $\mathcal{L}$tac2 lacks APIs for declaring Rocq inductive
types or scripting the vernacular language.

\paragraph{Metacoq}
% The Metacoq project~\cite{sozeau:hal-02167423} includes
% many components one of which is a description of Rocq's terms as a Rocq
% inductive faithful to the internal Rocq term data type, including De Bruijn
% indexes. A meta program is essentially a Rocq term in the TemplateMonad that
% gives programs access to some APIs to read/write from/to the logical
% environment and print messages. The unique feature of Metacoq is that it
% enables reasoning on ``meta'' programs. For example one can prove, in Rocq,
% that a meta program produces a term of a given type. Still it lacks APIs and
% formal definitions to deal with unification variables, hence proving properties
% about meta programs that manipulate incomplete syntax trees is out of reach.
The Metacoq project~\cite{sozeau:hal-02167423} encompasses several components,
one of which is a description of Rocq's terms as an inductive type, faithfully
reflecting the internal Rocq term data type, including De Bruijn indices. A
meta-program in Metacoq is essentially a Rocq term running in the
TemplateMonad, which allows access to various APIs to read from and write to
the logical environment. What sets Metacoq apart is its
ability to reason about "meta" programs. For instance, one can prove within
Rocq that a meta-program produces a term of a given type. However, Metacoq
currently lacks APIs and formal definitions to handle unification variables,
which limits the ability to prove properties about meta-programs that
manipulate incomplete syntax trees.

\paragraph{Mtac2}
% The Mtac2 tactic language~\cite{10.1145/3236773} is
% not actively developed anymore but has some interesting characteristics
% worth mentioning. It is a functional language but unlike Metacoq and $\mathcal{L}$tac2
% manages to completely hide De Bruijn indexes to the programmer.
% In particular its nu operator, used to cross binders, is closely related to
% Higher Order Abstract Syntax (see for example the pi operator of $\lambda$Prolog or
% the nab one of MLTS~\cite{10.1145/3354166.3354177}).
% Its type discipline is configurable and also concerns the object
% language ranging from dynamically typed to strongly typed.
% Finally it exposes the (higher order) unification of the objects language
% in the pattern matching construct of of the programming language.

The Mtac2 tactic language~\cite{10.1145/3236773} is no longer actively
developed but has some interesting features worth highlighting. Unlike Metacoq
and $\mathcal{L}$tac2, Mtac2 is a functional language that successfully hides
De Bruijn indices from the programmer. In particular the \texttt{nu} operator, used for
crossing binders, gives the comfort of HOAS and indeed
resembles $\lambda$Prolog's \texttt{pi} operator and MLTS's \texttt{nab}
operator~\cite{10.1145/3354166.3354177}.

Mtac2 offers a configurable type discipline that extends to the object
language, ranging from dynamically typed to strongly typed.
Additionally, it exposes the (higher-order) unification of the object
language through the language's pattern matching construct.

\paragraph{OCaml}
%All the languages implemented above are
%implemented in OCaml, as well as Rocq. It is possible to extend
%Rocq by writing an OCaml plugin but it is considerably harder.
%Still, if execution speed is more important that development time,
%this option can be considered.
%
While it is indeed possible to extend Rocq by writing an
OCaml plugin, this approach is considerably more challenging compared to using
higher-level languages like the ones mentioned above. Writing plugins in OCaml requires a deep
understanding of the internal workings of Rocq, and the development process is
more time-consuming.

However, if execution speed is a critical concern and the performance of the
system is a higher priority than development time, using OCaml plugins might be
a suitable choice.

\newpage

\printbibliography

\end{document}